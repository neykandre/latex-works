\begin{problem}
    Consider relation R(A,B,C,D) with the following functional dependencies: \\
    F = {A→D, AB→ C, AC→ B} \\
    A) What are all candidate keys? \\
    B) Convert R into 3NF using synthesis algorithm from textbook.
\end{problem}
\begin{solution} \
    \subsubsection*{Кандидаты:}
    Построим замыкания для наборов атрибутов:
    \begin{itemize}
        \item \((A)+ = \{ A, D \} \)
        \item \( (B)+ = \{ B \} \)
        \item \( (C)+ = \{ C \} \)
        \item \( (D)+ = \{ D \} \)
        \item \((AB)+ = \{ A, B, C, D \} \)
        \item \((AC)+ = \{ A, B, C, D \} \)
        \item \((AD)+ = \{ A, D \} \)
        \item \((BC)+ = \{ B, C \} \)
        \item \((BD)+ = \{ B, D \} \)
        \item \((CD)+ = \{ C, D \} \)
    \end{itemize}
    Нашли два кандидата: \( AB \) и \( AC \). Тройки, содержащие эти атрибуты рассматривать смысла нет, так как они не будут минимальными, а следовательно не будут кандидатами на ключ.
    \begin{itemize}
        \item \( (BCD)+ = \{ B, C, D \} \)
    \end{itemize}
    В замыкании этой оставшейся тройки не содержатся все атрибуты, а следовательно она не кандидат. Четверка ABCD не кандидат, так как она не минимальна. \\
    Итого кандидаты: $AB, AC$
    \subsubsection*{Конвертация в 3НФ:}
    F уже является минимальным базисом, так как все зависимости минимальны. \\
    Для каждой зависимости заведем отношение: 
    \begin{itemize}
        \item \( \mathcal{R}_A = AD \)
        \item \( \mathcal{R}_{AB} = ABC \)
        \item \( \mathcal{R}_{AC} = ABC \)
    \end{itemize}
    Заметим, что мы уже завели отношение для каждого из ключей \(AB, AC\). \\
    Видим, что \(\mathcal{R}_{AC} \subseteq \mathcal{R}_{AB}\). Тогда уберем отношение \( \mathcal{R}_{AC} \). \\
    Получили разложение \( \mathcal{R}_{A}, \mathcal{R}_{AC} \), каждое из которых удовлетворяет условиям 3НФ.
\end{solution}