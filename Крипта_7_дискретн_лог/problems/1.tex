\begin{problem}
    Используя алгоритм Силвера-Полига-Хеллмана, найти дискретный логарифм числа 123 по основанию 2 в \( \mathbb{F}_{181}^* \) (2 -- порождающий элемент в \( \mathbb{F}_{181}^* \))
\end{problem}
\begin{solution}
    Пусть \(q = 181, \ c = 123, \ \alpha = 2 \). Надо найти такое \(m\), что \(\alpha^m = c\). \\
    Разложим \(q - 1\):
    \[
    q - 1 = 180 = 2^2 \cdot 3^2 \cdot 5
    \]
    \begin{enumerate}
        \item Найдем \(m^{(1)} = m \ (\text{mod} \  2^2) \) \\
        Пусть \( m^{(1)} = m_0^{(1)} + 2m_1^{(1)} \) \\
        Обозначим \( \omega = \alpha^{\tfrac{q - 1}{2}} = 2^{\tfrac{180}{2}} = 2^{90} = 180 \). Тогда множество \(\Omega = \{ 1, \omega \} \) выглядит следущим образом:
        \[
        \Omega = \{ 1, 180 \}
        \]
        Найдем \(c^{\tfrac{q - 1}{2}} \):
        \[
        c^{\tfrac{q - 1}{2}} = 123^{\tfrac{180}{2}} = 123^{90} = 180 = \omega^1
        \]
        \[
        \Downarrow
        \]
        \[
        m_0^{(1)} = 1
        \]
        Теперь берём \( c_1 = c \cdot \alpha^{q - 1 - m_0^{(1)}} = 123 \cdot 2^{179} = 152\) \\
        Найдем \( c_1^{\tfrac{q - 1}{2^2}} \):
        \[
        c_1^{\tfrac{q - 1}{2^2}} = 152^{\tfrac{180}{2^2}} = 152^{45} = 180 = \omega^1
        \]
        \[
        \Downarrow
        \]
        \[
        m_1^{(1)} = 1
        \]
        Получили \(m^{(1)} = 1 + 2 \cdot 1 = 3\)

        \item Найдем \(m^{(2)} = m \ (\text{mod} \  3^2) \) \\
        Пусть \( m^{(2)} = m_0^{(2)} + 3m_1^{(2)} \) \\
        Обозначим \( \omega = \alpha^{\tfrac{q - 1}{3}} = 2^{\tfrac{180}{3}} = 2^{60} = 48, \ w^2 = 48^2 = 132 \). Тогда множество \(\Omega = \{ 1, \omega, \omega^2 \} \) выглядит следущим образом:
        \[
        \Omega = \{ 1, 48, 132 \}
        \]
        Найдем \(c^{\tfrac{q - 1}{3}} \):
        \[
        c^{\tfrac{q - 1}{3}} = 123^{\tfrac{180}{3}} = 123^{60} = 48 = \omega^1
        \]
        \[
        \Downarrow
        \]
        \[
        m_0^{(2)} = 1
        \]
        Теперь берём \( c_1 = c \cdot \alpha^{q - 1 - m_0^{(2)}} = 123 \cdot 2^{179} = 152\) \\
        Найдем \( c_1^{\tfrac{q - 1}{3^2}} \):
        \[
        c_1^{\tfrac{q - 1}{3^2}} = 152^{\tfrac{180}{3^2}} = 152^{20} = 48 = \omega^1
        \]
        \[
        \Downarrow
        \]
        \[
        m_1^{(2)} = 1
        \]
        Получили \(m^{(2)} = 1 + 3 \cdot 1 = 4\)
        \item Найдем \(m^{(3)} = m \ (\text{mod} \  5) \) \\
        Обозначим \( \omega = \alpha^{\tfrac{q - 1}{5}} = 2^{\tfrac{180}{5}} = 2^{36} = 59, \ \omega^2 = 59^2 = 42, \ w^3 = 59^3 = 125, \ w^4 = 59^4 = 135 \). Тогда множество \(\Omega = \{ 1, \omega, \omega^2, \omega^3, \omega^4 \} \) выглядит следущим образом:
        \[
        \Omega = \{ 1, 59, 42, 125, 135 \}
        \]
        Найдем \(c^{\tfrac{q - 1}{5}} \):
        \[
        c^{\tfrac{q - 1}{5}} = 123^{\tfrac{180}{5}} = 123^{36} = 135 = \omega^4
        \]
        \[
        \Downarrow
        \]
        \[
        m^{(3)} = 4
        \]
        Получили \(m^{(3)} = 4\)
    \end{enumerate}
    Составим систему сравнений:
    \[
    \begin{cases}
        m = m^{(1)} = 3 \quad (\text{mod} \ 2^2) \\
        m = m^{(2)} = 4 \quad (\text{mod} \ 3^2) \\
        m = m^{(3)} = 4 \quad (\text{mod} \ 5)
    \end{cases}
    \Longleftrightarrow
    \begin{cases}
        m = 3 \quad (\text{mod} \ 4) \\
        m = 4 \quad (\text{mod} \ 9) \\
        m = 4 \quad (\text{mod} \ 5)
    \end{cases}
    \Longleftrightarrow
    m = 139 \quad (\text{mod} \ 180)
    \]
    Получили, что \( 2^{139} = 123 \ (\text{mod} \ 181) \).
\end{solution}