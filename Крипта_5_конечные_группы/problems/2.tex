\begin{problem}
    Проверить, является ли факторкольцо \(\mathbb{Z}_5[x] / \langle x^4 + 2x^2 + 3 \rangle \) полем. Если да, то сколько в нём элементов? Если нет, показать, почему это не поле.
\end{problem}
\begin{solution}
    \(\mathbb{Z}_5[x] / \langle x^4 + 2x^2 + 3 \rangle \) -- поле \(\Longleftrightarrow x^4 + 2x^2 + 3 \) неприводимый над \(\mathbb{Z}_5\). \\
    Попробуем разложить в произведение многочленов первой и третей степени. Если раскладывается на многочлен первой степени, то есть корни.
    \begin{itemize}
        \item \( x = 0: \quad 0 + 0 + 3 = 3 \neq 0\)
        \item \( x = 1: \quad 1 + 2 + 3 = 1 \neq 0\)
        \item \( x = 2: \quad 16 + 8 + 3 = 1 + 3 + 3 = 7 = 2 \neq 0\)
        \item \( x = 3: \quad 3^4 + 2 \cdot 3^2 + 3 = (-2)^4 + 2 \cdot (-2)^2 + 3 = 2 \neq 0 \)
        \item \( x = 4: \quad 4^4 + 2 \cdot 4^2 + 3 = (-1)^4 + 2 \cdot (-1)^2 + 3 = 1 \neq 0\)
    \end{itemize}
    Нет корней, значит не раскладывается произведение многочленов первой и третей степени.
    Остается проверить раскладывается ли \( x^4 + 2x^2 + 3 \) на произведение двух неприводимых многочленов второй степени. \\
    Пусть: \[ (ax^2 + bx + c)(dx^2 + ex + f) = x^4 + 2x^2 + 3 \]
    Б.О.О. положим \(a = 1\) (можно всегда вынести коэффициент из первого многочлена и занести во второй).
    \[ (x^2 + bx + c)(dx^2 + ex + f) = x^4 + 2x^2 + 3 \]
    \[ dx^4 + (e + bd)x^3 + (f + eb + cd)x^2 + (ce + bf)x + fc = x^4 + 2x^2 + 3 \]
    \[ \begin{cases}
        d = 1 \\
        e + bd = 0 \\
        f + eb + cd = 2 \\
        ce + bf = 0 \\
        fc = 3
    \end{cases} \Longleftrightarrow
    \begin{cases}
        d = 1 \\
        e + b = 0 \\
        f + eb + c = 2 \\
        ce + bf = 0 \\
        fc = 3
    \end{cases} \Longleftrightarrow
    \begin{cases}
        d = 1 \\
        e = -b = 4b \\
        f + 4b^2 + c = 2 \\
        -bc + bf = 0 \\
        fc = 3
    \end{cases} \Longleftrightarrow
    \begin{cases}
        d = 1 \\
        e = 4b \\
        f + 4b^2 + c = 2 \\
        b(f - c) = 0 \\
        fc = 3
    \end{cases} \Longleftrightarrow
    \]
    \[
    \Longleftrightarrow
    \left[
    \begin{array}{l}
        \begin{cases}
            d = 1 \\
            b = 0 \\
            e = 0 \\
            f + c = 2 \\
            fc = 3
        \end{cases} \\
        \begin{cases}
            d = 1 \\
            f = c \\
            e = 4b \\
            2c + 4b^2 = 2 \\
            c^2 = 3
        \end{cases}
    \end{array}
    \right.
    \Longleftrightarrow
    \left[
    \begin{array}{l}
        \begin{cases}
            d = 1 \\
            b = 0 \\
            e = 0 \\
            f = 2 + 4c\\
            c(2 + 4c) = 3
        \end{cases} \\
        \begin{cases}
            d = 1 \\
            f = c \\
            e = 4b \\
            2c + 4b^2 = 2 \\
            c^2 = 3
        \end{cases}
    \end{array}
    \right.
    \Longleftrightarrow
    \left[
    \begin{array}{l}
        \begin{cases}
            d = 1 \\
            b = 0 \\
            e = 0 \\
            f = 2 + 4c\\
            c(1 + 2c) = 4
        \end{cases} \\
        \begin{cases}
            d = 1 \\
            f = c \\
            e = 4b \\
            2c + 4b^2 = 2 \\
            c^2 = 3
        \end{cases}
    \end{array}
    \right.
    \Longleftrightarrow
    \left[
    \begin{array}{l}
        \begin{cases}
            d = 1 \\
            b = 0 \\
            e = 0 \\
            f = 2 + 4c\\
            2c^2 + c + 1 = 0
        \end{cases} \\
        \begin{cases}
            d = 1 \\
            f = c \\
            e = 4b \\
            2c + 4b^2 = 2 \\
            c^2 = 3
        \end{cases}
    \end{array}
    \right.
    \]
    Либо должно выполняться \( 2c^2 + c + 1 = 0 \), либо \( c^2 = 3 \). \\
    \( 2c^2 + c + 1 = 0 \Longleftrightarrow b^2 - 4ac =1 - 4 \cdot 2 \cdot 1 = 1 + 2 = 3\) -- квадратичный вычет \\
    То есть два варианта сводятся к вопросу, является ли 3 квадратичным вычетом по модулю 5. \\
    Проверим по критерию Эйлера: \( 3^{\frac{5 - 1}{2}} = 3^2 = 9 = -1 \Longleftrightarrow\) 3 не квадратичный вычет по модулю 5. \\
    Значит система не имеет решений, то есть \( x^4 + 2x^2 + 3 \) неприводимый над \(\mathbb{Z}_5\). \\
    \(\mathbb{Z}_5[x] / \langle x^4 + 2x^2 + 3 \rangle \) -- поле.
\end{solution}