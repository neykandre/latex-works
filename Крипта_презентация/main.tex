\documentclass{beamer}

% Установка темы презентации
\usetheme{Madrid}

% Подключение пакетов
\usepackage[utf8]{inputenc} % Кодировка UTF-8
\usepackage[russian]{babel} % Русский язык
\usepackage{amsmath, amssymb} % Математические формулы
\usepackage{graphicx} % Работа с графикой

% Информация о презентации
\title[Математика разделенного секрета]{Математика разделенного секрета:\\ Пороговые (n,k)-схемы доступа,\\ схема Шамира и схема Блэкли}
\author[Носов Андрей]{Носов Андрей БПИ-232}
\date{\today}
\institute{}

\begin{document}

% Титульный слайд
\begin{frame}
    \titlepage
\end{frame}

% % Содержание
% \begin{frame}{Содержание}
%     \tableofcontents
% \end{frame}

% Раздел 1: Схема разделения секрета
\section{Схема разделения секрета}
\begin{frame}{Определение}
    \begin{itemize}
        \item Схема разделения секрета (СРС) — это криптографический протокол, позволяющий разделить секрет \( S \) на \( n \) долей \( S_1, S_2, \dots, S_n \), так что:
        \begin{itemize}
            \item Любая подгруппа участников размером \( k \) или более может восстановить секрет.
            \item Любая подгруппа из менее чем \( k \) участников ничего не знает о секрете.
        \end{itemize}
    \end{itemize}
\end{frame}

% Раздел 2: Основные лица
\section{Основные лица}
\begin{frame}{Участники и дилер}
    \begin{itemize}
        \item **Дилер** — доверенное лицо, которое:
        \begin{itemize}
            \item Генерирует секрет \( S \).
            \item Вычисляет \( n \) долей \( S_1, S_2, \dots, S_n \).
            \item Передаёт доли участникам.
        \end{itemize}
        \item **Участники** — лица, получающие доли секрета. Они объединяются для восстановления секрета.
    \end{itemize}
\end{frame}

% Раздел 3: Sharing and Reconstruction
\section{Sharing and Reconstruction}
\begin{frame}{Функции разделения и восстановления секрета}
    \begin{itemize}
        \item **Функция разделения**:
        \[
        \text{Share}(S) \to \{S_1, S_2, \dots, S_n\}
        \]
        Разбивает секрет \( S \) на \( n \) долей.
        \item **Функция восстановления**:
        \[
        \text{Reconstruct}(\{S_1, S_2, \dots, S_k\}) \to S
        \]
        Объединяет \( k \) долей для получения оригинального секрета.
    \end{itemize}
\end{frame}

% Раздел 4: Идеальная СРС
\section{Идеальная СРС}
\begin{frame}{Определение идеальной СРС}
    \begin{itemize}
        \item Идеальная схема разделения секрета:
        \begin{itemize}
            \item Доля каждого участника имеет тот же размер, что и секрет \( S \).
            \item Участники не обладают избыточной информацией.
        \end{itemize}
    \end{itemize}
\end{frame}

% Раздел 5: Совершенная СРС
\section{Совершенная СРС}
\begin{frame}{Определение}
    \begin{itemize}
        \item Совершенная СРС:
        \[
        P(S | \text{меньше чем } k \text{ долей}) = P(S)
        \]
        \item Свойство «всё или ничего»: 
        \begin{itemize}
            \item Меньше \( k \) долей — нет информации о секрете.
            \item \( k \) или больше долей — секрет полностью восстанавливается.
        \end{itemize}
    \end{itemize}
\end{frame}

% Раздел 6: (n,k)-пороговая СРС
\section{(n,k)-пороговая СРС}
\begin{frame}{Определение}
    \begin{itemize}
        \item \( (n, k) \)-пороговая схема:
        \begin{itemize}
            \item Секрет делится между \( n \) участниками.
            \item Для восстановления секрета требуется \( k \) участников (\( k \leq n \)).
        \end{itemize}
    \end{itemize}
\end{frame}

% Раздел 7: Пример (n,n)-пороговой СРС
\section{Пример с (n,n)-пороговой СРС}
\begin{frame}{Пример}
    \begin{itemize}
        \item Секрет делится на \( n \) долей, каждая из которых равна \( S \).
        \item Доказательство:
        \begin{itemize}
            \item \( n \) участников объединяются и полностью восстанавливают секрет.
            \item Менее \( n \) участников ничего не знают о секрете.
        \end{itemize}
    \end{itemize}
\end{frame}

% Раздел 8: Полином Лагранжа
\section{Полином Лагранжа}
\begin{frame}{Восстановление по \( k \) точкам}
    \begin{itemize}
        \item Полином степени \( k-1 \):
        \[
        f(x) = \sum_{i=1}^{k} y_i \prod_{j \neq i} \frac{x - x_j}{x_i - x_j}
        \]
        \item Используется для восстановления секрета \( f(0) \) по \( k \) точкам.
    \end{itemize}
\end{frame}

% Раздел 9: Схема Шамира
\section{Схема Шамира}
\begin{frame}{Описание}
    \begin{itemize}
        \item Генерация случайного полинома степени \( k-1 \):
        \[
        f(x) = a_0 + a_1x + \dots + a_{k-1}x^{k-1}
        \]
        где \( a_0 \) — секрет.
        \item Доли: точки \( (x_i, f(x_i)) \), \( x_i \neq 0 \).
        \item Доказательства:
        \begin{itemize}
            \item Совершенность: \( k-1 \) долей не дают информацию о \( a_0 \).
            \item Идеальность: размер долей равен размеру секрета.
        \end{itemize}
    \end{itemize}
\end{frame}

% Раздел 10: Схема Блэкли
\section{Схема Блэкли}
\begin{frame}{Описание}
    \begin{itemize}
        \item Использует систему линейных уравнений:
        \[
        A \cdot X = B
        \]
        где \( A \) — матрица коэффициентов, \( X \) — вектор секретов, \( B \) — вектор долей.
        \item Пример для 3D:
        \includegraphics[width=0.8\textwidth]{3d-example.png} % Вставьте подходящую картинку
    \end{itemize}
\end{frame}

% Заключение
\section{Заключение}
\begin{frame}{Заключение}
    \begin{itemize}
        \item СРС — важный инструмент для безопасного хранения данных.
        \item Различные схемы применяются в зависимости от задач.
        \item Перспективы: улучшение устойчивости и эффективность вычислений.
    \end{itemize}
\end{frame}

% Последний слайд
\begin{frame}
    \centering
    \Huge Спасибо за внимание!
\end{frame}

\end{document}
