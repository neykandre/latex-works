\documentclass{beamer}

% Установка темы презентации
\usetheme{Madrid}

% Подключение пакетов
\usepackage[utf8]{inputenc} % Кодировка UTF-8
\usepackage[english,russian]{babel}
\usepackage{amsmath, amssymb}

% Настройка окружения для определений
\newtheorem{define}{Определение}
\newtheorem{Note}{Замечание}
\newtheorem{claim}{Утверждение}
\setbeamertemplate{caption}{\raggedright\insertcaption\par}

% Информация о презентации
\title[Математика разделенного секрета]{Математика разделенного секрета:\\ Пороговые (n,k)-схемы доступа,\\ схема Шамира и схема Блэкли}
\author[Носов Андрей]{Носов Андрей БПИ-232}
\date{}
\institute{}

\begin{document}

% Титульный слайд
\begin{frame}
    \titlepage
\end{frame}

% % Содержание
% \begin{frame}{Содержание}
%     \tableofcontents
% \end{frame}

% Раздел 1: Схема разделения секрета
\section{Схема разделения секрета}
\begin{frame}{Схема разделения секрета}
    Пусть есть секрет и группа людей, между которыми нужно распределить этот секрет. Задача состоит в следующем. \\Нужно придумать, какую информацию предоставить каждому человеку, чтобы только <<разрешенные>> подгруппы могли восстановить секрет.
    \begin{define}
        \textbf{Схема разделения секрета (СРС)} --- способ распределения секрета среди группы участников, в котором
        \begin{itemize}
            \item каждому из участников достается своя некая \textbf{доля};
            \item только разрешенные подмножества группы участников могут восстановить секрет.
        \end{itemize}
    \end{define}
\end{frame}

% Раздел 2: Основные лица
\begin{frame}{Участники и дилер}
    \begin{define}
        В СРС выделяются следующие роли:
        \begin{itemize}
            \item \textbf{Дилер} --- доверенное лицо, которое
            \begin{itemize}
                \item Знает секрет \( s_0 \);
                \item Вычисляет \( n \) долей \( s_1, s_2, \dots, s_n \);
                \item Передает доли участникам.
            \end{itemize}
            \item \textbf{Участники} --- лица, получающие доли секрета. Они объединяются для восстановления секрета.
        \end{itemize}
    \end{define}
\end{frame}

\begin{frame}{Математическая модель}
    Зададим модель СРС следующим набором:
    \begin{itemize}
        \item Множества \( S_0, S_1, \dots, S_n \). \\
        \(S_0\) --- множество секретов. \\ 
        \(S_i \ (i = \overline{1, n})\) --- множество долей \(i\)-го участника.
        \item Распределение вероятностей \(P\) на их декартовом произведении \(S = S_0 \times S_1 \times \dots \times S_n\).\\
        Соответствующие случайные величины обозначим \(\xi_i\).
        \item Множество \(\mathcal{A} \subseteq 2^{\underline{n}}\), называемое структурой доступа. \\Тогда любое множество \(A \in \mathcal{A}\) задает разрешенное подмножество группы участников.
    \end{itemize} 
\end{frame}

\begin{frame}{Замечания}
    \begin{define}
        Участник \(x \in \{1, \dots, n\}\) называется \textbf{несущественным} для структуры доступа \(\mathcal{A}\), если 
        \begin{itemize}
            \item \(\forall A \not\in \mathcal{A} \Longrightarrow A \cup x \not\in \mathcal{A}\).
        \end{itemize}
    \end{define}
    \begin{Note}
        Очевидно, что несущественные участники настолько несущественны для разделения секрета, что им просто не нужно посылать никакой информации. Поэтому можно рассматривать только такие структуры доступа \(\mathcal{A}\), для которых все элементы являются существенными.
    \end{Note}
    \begin{Note}
        Естественно полагать, что \(\mathcal{A}\) является монотонной структурой, т. е. \( \left.
        \begin{array}{c}
            A \subseteq B \\
            A \in \mathcal{A}
            \end{array} \right\} \Longrightarrow B \in \mathcal{A}\)
    \end{Note}
\end{frame}

% Раздел 3: Sharing and Reconstruction
\section{Sharing and Reconstruction}
\begin{frame}{Функции разделения и восстановления секрета}
    В такой математической модели можно легко задать функции разделения и восстановления секрета.
    \begin{define}
        Функция разделения:
        \[
        \text{Share}: \ S_0 \to (S_1 \times \dots \times S_n)
        \]
    \end{define}
    \begin{define}
        Функция восстановления:
        \[
        \text{Reconstruct}: \ (S_{i_1} \times \dots \times S_{i_m}) \to S_0
        \]
    \end{define}
\end{frame}

\begin{frame}{Примитивная идея}
    \begin{figure}
    \centering
    \includegraphics[width=0.65\linewidth]{2024_11_26_0hx_Kleki.png}
\end{figure}
\end{frame}

% Раздел 5: Совершенная СРС
\section{Совершенная СРС}
\begin{frame}{Совершенная СРС}
    \begin{define}
        СРС, реализующая структуру доступа \(\mathcal{A}\), называется \textbf{совершенной}, если
        \begin{itemize}
            \item \(P(\xi_0 = x \ \vline \ \xi_i = s_i, i \in A) \in \{ 0, 1 \}\) \quad для \(A \in \mathcal{A}\)
            \item \(P(\xi_0 = x \ \vline \ \xi_i = s_i, i \in A) = P(\xi_0 = x)\) \quad для \(A \not\in \mathcal{A}\)
        \end{itemize}
    \end{define}
    Это определение можно понять следующим образом:
    \begin{itemize}
        \item участники из разрешенного множества \(A\) (\(A \in \mathcal{A}\)) вместе могут однозначно восстановить значение секрета;
        \item участники, образующие неразрешенное множество \(A\) (\(A \not\in \mathcal{A}\)),не получают никакой дополнительной информации о секрете. Т. е. вероятность того, что значение секрета \(\xi_0 = x\), не зависит от значений долей \(\xi_i, i \in A\). 
    \end{itemize}
    Особый интерес с точки зрения безопасности вызывают СРС, обладающие данным свойством. Поэтому далее будем рассматривать только их.
\end{frame}

% Раздел 4: Идеальная СРС
\section{Идеальная СРС}
\begin{frame}{Идеальная СРС}
    \begin{define}
        \textbf{Идеальная СРС} — это схема, в которой доля каждого участника имеет тот же размер, что и секрет:
        \[
        |S_i| = |S_0|, \ \forall i = \overline{1, n}
        \]
    \end{define}
\end{frame}

\section{Простейшая структура доступа}
\begin{frame}{Простейшая структура доступа}
    Рассмотрим такую СРС, что только все участники вместе могут восстановить секрет, т. е. \(\mathcal{A} = \{ \{1, \dots, n \} \}\). \\
    Пусть \(S_0 = S_1 = \dots = S_n = \mathbb{Z}_p\). \\
    \begin{itemize}
        \item Share: \\
        Дилер генерирует значения долей для первых \(n - 1\) участников: \(s_1, \dots, s_{n - 1}\). \\
        \(s_n\) вычисляется: \(s_n = s_0 - s_1 - \dots - s_{n - 1}\).
        \item Reconstruct: \\
        \(s_0 = s_1 + \dots + s_{n - 1} + s_n\)
    \end{itemize}
\end{frame}

\begin{frame}{Простейшая структура доступа}
    \begin{claim}
        Описанная СРС является идеальной и совершенной.
    \end{claim}
    Доказательство: Идеальность очевидна, т. к. \(S_0 = S_1 = \dots = S_n = \mathbb{Z}_p\) \\
    Докажем совершенность. Нужно:
    \begin{itemize}
        \item \(P(\xi_0 = x \ \vline \ \xi_1 = s_1, \ \xi_2 = s_2, \dots, \xi_n = s_n) \in \{0,1\}\) \\
        \item \(P(\xi_0 = x \ \vline \ \xi_{i_1} = s_{i_1}, \dots, \xi_{i_k} = s_{i_k}) = P(\xi_0 = X)\) при \(k < n\)
    \end{itemize}
\end{frame}

\begin{frame}{Простейшая структура доступа}
\begin{itemize}
    \item 
    \(P(\xi_0 = x \ \vline \ \xi_1 = s_1, \ \xi_2 = s_2, \dots, \xi_n = s_n) \in \{0, 1\}\)
\end{itemize} \\ \ \\
    Доказательство:
    \[
        P(\xi_0 = x \ \vline \ \xi_1 = s_1, \ \xi_2 = s_2, \dots, \xi_n = s_n) = 
        \]
        \[
        = P(\xi_0 = x \ \vline \ \xi_1 = s_1, \ \xi_2 = s_2, \dots, \xi_0 - \xi_1 - \dots - \xi_{n - 1} = s_n) =
        \]
        \[
        = P(x - s_1 - \dots - s_{n - 1} = s_n) = P(x = s_1 + \dots + s_n)
        \]
        Если \(s_1 + \dots + s_n = x\), то \(P = 1\); \\
        Если \(s_1 + \dots + s_n \neq x\), то \(P = 0\). \\
\end{frame}

\begin{frame}{Простейшая структура доступа}
    \begin{itemize}
        \item \(P(\xi_0 = x \ \vline \ \xi_{i_1} = s_{i_1}, \dots, \xi_{i_k} = s_{i_k}) = P(\xi_0 = x)\) при \(k < n\)
    \end{itemize}
    \\ \ \\
    Доказательство: Для этого докажем, что \(P(\xi_i = x) = \frac{1}{p}, \ \forall i = \overline{0, n}\). Очевидно, что \(P(\xi_i = x) = \frac{1}{p}, \ \forall i = \overline{0, n - 1}\). \\Для \(\xi_n\):
    \[P(\xi_n = x) = P(\xi_0 - \xi_1 - \dots - \xi_n = x) = P(\xi_0 = \xi_1 + \dots + \xi_{n-1} + x) = \]
    \[
    = \sum_{s_i \in \mathbb{Z}_p} P(\xi_1 = s_1, \dots, \xi_{n - 1} = s_{n-1}, \ \xi_0 = s_1 + \dots + s_{n -1 } + x) = 
    \]
    \[
    = \sum_{s_i \in \mathbb{Z}_p} P(\xi_1 = s_1, \dots, \xi_{n - 1} = s_{n-1}) P(\xi_0 = s_1 + \dots + s_{n -1 } + x) = 
    \]
    \[
    = \frac{1}{p} \sum_{s_i \in \mathbb{Z}_p} P(\xi_1 = s_1, \dots, \xi_{n - 1} = s_{n-1}) = \frac{1}{p} \cdot 1 = \frac{1}{p}
    \]
\end{frame}

\begin{frame}{Простейшая структура доступа}
Теперь нужно доказать, что \(\forall i = \overline{1, n} \ \ \xi_0, \xi_1, \dots, \xi_{i - 1}, \xi_{i + 1}, \dots, \xi_n\) независимы в совокупности. \\
Очевидно, что \(\xi_0, \xi_1, \dots, \xi_{n - 1}\) независимы по построению. \\
Необходимо доказать, что \(\forall i = \overline{1, n - 1} \ \ \xi_0, \xi_1, \dots, \xi_{i - 1}, \xi_{i + 1}, \dots, \xi_n\) независимы.
    \[
    P(\xi_0 = s_0, \xi_1 = s_1, \dots, \xi_{i - 1} = s_{i - 1}, \xi_{i + 1} = s_{i + 1}, \dots, \xi_n = s_n) = 
    \]
    \[
    = P(\dots, \xi_0 - \xi_1 - \dots - \xi_{i - 1} - \xi_i - \xi_{i + 1} - \dots - \xi_{n - 1} = s_n) = 
    \]
    \[
    = P(\dots, s_0 - s_1 - \dots - s_{i - 1} - \xi_i - s_{i + 1} - \dots - s_{n - 1} = s_n) = 
    \]
    \[
    = P(\dots, \xi_i = s_0 - s_1 - \dots - s_{i - 1} - s_{i + 1} - \dots - s_n) = \left(\frac{1}{p}\right)^{n} = 
    \]
    \[
    = P(\xi_0 = s_0) P(\xi_1 = s_1) \dots P(\xi_{i - 1} = s_{i - 1}) P(\xi_{i + 1} = s_{i + 1}) \dots  P(\xi_{n} = s_{n}) 
    \]
    
\end{frame}

\begin{frame}{Простейшая структура доступа}
Так как \(\forall i = \overline{1, n} \ \ \xi_0, \xi_1, \dots, \xi_{i - 1}, \xi_{i + 1}, \dots, \xi_n\) независимы, то при \(k < n\)
    \[
    P(\xi_0 = x \ | \ \xi_{i_1} = s_{i_1}, \dots, \xi_{i_k} = s_{i_k}) = P(\xi_0 = x)
    \]
\end{frame}

% Раздел 6: (n,k)-пороговая СРС
\section{(n,k)-пороговая СРС}
\begin{frame}{(n,k)-пороговая СРС}
    \begin{define}
        \( (n, k) \)-пороговая схема разделения секрета:
        \begin{itemize}
            \item \(n\) участников;
            \item \(\mathcal{A} = \{ K \subseteq \{1, \dots, n\} \ \vline \ |K| \geq k \}\).
        \end{itemize}
    \end{define}
    То есть любые \(k\) участников, собравшись вместе, могут восстановить секрет, а любые \(k - 1\) участников не получат никакой дополнительной информации о секрете. \\
    \begin{Note}
        Описанная ранее простейшая структура доступа соответствует \((n, n)\)-пороговой СРС.
    \end{Note}
\end{frame}

\begin{frame}{Начало}
    \begin{itemize}
        \item История СРС начинается с 1979 года, когда Ади Шамир и Джордж Блэкли независимо друг от друга предложили методы того, как составлять (n,k)-пороговые СРС.
    \end{itemize}
    \begin{figure}
    \begin{minipage}{.5\textwidth}
        \centering
        \includegraphics[width=0.612\linewidth]{shamir.jpg}
        \caption{Ади Шамир}
    \end{minipage}%
    \begin{minipage}{.5\textwidth}
        \centering
        \includegraphics[width=0.7\linewidth]{blakley.jpg}
        \caption{Джордж Блэкли}
    \end{minipage}
    \end{figure}
    
\end{frame}

% Раздел 9: Схема Шамира
\section{Схема Шамира}
\begin{frame}{Схема Шамира}
    Ади Шамир предложил следующую СРС. \\
    Сопоставим участникам \(n\) различных чисел \(x_1, \dots, x_n \in \mathbb{F}_q\) и положим \(x_0 = 0\).
    \begin{itemize}
        \item Share: \\
        Дилер генерирует \(k - 1\) чисел \(a_i, i = \overline{1, k - 1}\). \(a_0\) полагается равным \(s_0\). Составляет многочлен:
        \[
        f(x) = a_0 + a_1x + \dots + a_{k - 1}x^{k - 1}
        \]
        И посылает \(i\)-му участнику его долю \(s_i = f(x_i)\).
        \item Reconstruct: \\
        Любые \(k\) участников собираются вместе и по \(k\) точкам восстанавливают многочлен \(f(x)\), т. к. \(\text{deg}(f) = k - 1\). \\Например, через интерполяционный многочлен Лагранжа или через решение СЛАУ. \\
        Затем находят \(s_0 = f(0)\).
        
    \end{itemize}
\end{frame}

\begin{frame}{Схема Шамира}
    Визуализация для \(k = 4\):
    \begin{figure}
    \centering
    \includegraphics[width=\linewidth]{shamir_sceme.png}
\end{figure}
\end{frame}

\begin{frame}{Схема Шамира. Пример}
    Интерполяционный многочлен Лагранжа степени не больше \(n\) по \(n + 1\) точкам: \[L_n(x) = \sum_{i = 0}^n y_i \left( \prod_{j = 0, j \neq i}^n \frac{x - x_j}{x_i - x_j}\right)\]
    Пример восстановления секрета:\\ Реализована \((n, 3)\)-пороговая СРС Шамира. \\
    Алисе соответсвует число 1, Бобу --- 2, Еве --- 3. \\
    Доля Алисы --- 2, доля Боба --- 3, доля Евы --- 5. \\
    Восстановление многочлена: \[L_2(x) = 2 \cdot \frac{x - 2}{1 - 2} \cdot \frac{x - 3}{1 - 3} + 3 \cdot \frac{x - 1}{2 - 1} \cdot \frac{x - 3}{2 - 3} + 5 \cdot \frac{x - 1}{3 - 1} \cdot \frac{x - 2}{3 - 2} \]
    Найдем секрет: \[
        L_2(0) = 2 \cdot \frac{-2}{-1} \cdot \frac{-3}{-2} + 3 \cdot \frac{-1}{1} \cdot \frac{-3}{-1} + 5 \cdot \frac{-1}{2} \cdot \frac{-2}{1} = 2
    \]
    
    
\end{frame}

\section{Идеальность}
\begin{frame}{Схема Шамира}
    \begin{claim}
        Схема Шамира является идеальной СРС.
    \end{claim}
    Доказательство: \(\forall i = \overline{1, n} \ |S_i| = |\mathbb{F}_q| = |S_0|\).
\end{frame}

\section{Совершенность}
\begin{frame}{Схема Шамира}
    \begin{claim}
        Схема Шамира является совершенной СРС.
    \end{claim}
    Доказательство: Нужно
    \begin{itemize}
        \item \(P(\xi_0 = x \ \vline \ \xi_{i_1} = s_{i_1}, \dots, \xi_{i_k} = s_{i_k}) \in \{ 0, 1 \} \); \\
        \item \(P(\xi_0 = x \ \vline \ \xi_{i_1} = s_{i_1}, \dots, \xi_{i_t} = s_{i_t}) = P(\xi_0 = x) \) при \(t < k\).
    \end{itemize}
\end{frame}

\begin{frame}{Схема Шамира}
    \begin{itemize}
        \item \(P(\xi_0 = x \ \vline \ \xi_{i_1} = s_{i_1}, \dots, \xi_{i_k} = s_{i_k}) \in \{ 0, 1 \} \)
    \end{itemize}
    Запишем систему уравнений относительно неизвестных \(a_0, \dots, a_{k - 1}\):
    \[
    \begin{cases}
        s_{i_1} = a_0 + a_1x_{i_1} + \dots + a_{k - 1}x_{i_1}^{k - 1} \\
        \vdots \\
        s_{i_k} = a_0 + a_1x_{i_k} + \dots + a_{k - 1}x_{i_k}^{k - 1} \\
    \end{cases}
    \]
    Перепишем в матричном виде:
    \[
    \begin{pmatrix}
        s_{i_1} \\ \vdots \\ s_{i_k}
    \end{pmatrix} = \begin{pmatrix}
        1 & x_{i_1} & \dots & x_{i_1}^{k - 1} \\
        \vdots & \vdots & \ddots & \vdots \\
        1 & x_{i_k} & \dots & x_{i_k}^{k - 1}
    \end{pmatrix} \begin{pmatrix}
        a_0 \\ \vdots \\ a_{k - 1}
    \end{pmatrix}
    \]
\end{frame}

\begin{frame}{Схема Шамира}
\begin{itemize}
        \item \(P(\xi_0 = x \ \vline \ \xi_{i_1} = s_{i_1}, \dots, \xi_{i_k} = s_{i_k}) \in \{ 0, 1 \} \)
    \end{itemize}
    \[
    \begin{pmatrix}
        s_{i_1} \\ \vdots \\ s_{i_k}
    \end{pmatrix} = \begin{pmatrix}
        1 & x_{i_1} & \dots & x_{i_1}^{k - 1} \\
        \vdots & \vdots & \ddots & \vdots \\
        1 & x_{i_k} & \dots & x_{i_k}^{k - 1}
    \end{pmatrix} \begin{pmatrix}
        a_0 \\ \vdots \\ a_{k - 1}
    \end{pmatrix}
    \]
    Определитель матрицы Вандермонда не равен 0 \(\Longrightarrow\) есть единственное решение \( \Longrightarrow\) можем найти это единственной решение \((a_0, \dots, a_{k - 1})^T\) \(\Longrightarrow\) можем найти секрет \(s_0 = a_0\). \\
    Тогда для рассматриваемой вероятности будет верно:
    \[
    P(\xi_0 = x \ | \ \xi_{i_1} = s_{i_1}, \dots, \xi_{i_k} = s_{i_k}) = P(\xi_0 = x \ | \ \xi_0 = s_0) = 
    \]
    \[
    = P(x = s_0) \in \{ 0, 1 \}
    \]
\end{frame}

\begin{frame}{Схема Шамира}
    \begin{itemize}
        \item \(P(\xi_0 = x \ \vline \ \xi_{i_1} = s_{i_1}, \dots, \xi_{i_t} = s_{i_t}) = P(\xi_0 = x) \) при \(t < k\).
    \end{itemize}
    Докажем, что \(P(\xi_0 = x \ \vline \ \xi_{i_1} = s_{i_1}, \dots, \xi_{i_{k - 1}} = s_{i_{k - 1}}) = P(\xi_0 = x)\). \\ \ \\
    Зная \(k - 1\) точек \((x_{i_1}, s_{i_1}), \dots, (x_{i_{k - 1}}, s_{i_{k - 1}})\), через которые проходит многочлен \(f(x)\) степени \(k - 1\), можно для каждого значения секрета \(s_0 = f(0)\) построить ровно один многочлен \(f(x)\). \\
    Значит даже зная, что \(\xi_{i_j} = s_{i_j}\), \ СВ \(\xi_0\) распределена равномерно на всем поле \(\mathbb{F}_q\).
    
\end{frame}

% Раздел 10: Схема Блэкли
\section{Схема Блэкли}
\begin{frame}{Схема Блэкли}
    Джордж Блэкли предложил следующую СРС. \\
    \begin{itemize}
        \item Share:
        Дилер генерирует \(k - 1\) число \(b_1, \dots, b_{k - 1} \in \mathbb{F}_q\) и задает точку в \(k\)-мерном пространстве с координатами \((s_0, b_1, \dots, b_{k - 1})\). \\Для каждого участника он составляет уравнение гиперплоскости, которая проходит через заданную точку. Для этого дилер для \(i\)-го участника генерирует \(k\) чисел \(a_{i_1}, \dots, a_{i_k} \in \mathbb{F}_q\).
        Так как уравнение плоскости имеет вид
        \(
        a_{i_1}x_1 + a_{i_2}x_2 + \dots + a_{i_k}x_k + d_i = 0
        \), то для каждого участника необходимо еще вычислить \(d_i\):
        \[
        \begin{smallmatrix}
             d_1 = - (a_{1_1}s_0 + a_{1_2}b_1 + \dots + a_{1_k}b_{k-1}) \\
             \vdots \\
             d_i = - (a_{i_1}s_0 + a_{i_2}b_1 + \dots + a_{i_k}b_{k-1}) \\
             \vdots \\
             d_n = - (a_{n_1}s_0 + a_{n_2}b_1 + \dots + a_{n_k}b_{k-1}) \\
        \end{smallmatrix}
        \]
        Доля \(i\)-го участника задается вектором \(s_i = (a_{i_1}, \dots, a_{i_k}, d_i)\).
    \end{itemize}
\end{frame}

\begin{frame}{Схема Блэкли}
    \begin{itemize}
        \item Reconstruct: \\
        Любые \(k\) участников собираются вместе и по уравнениям \(k\) плоскостей однозначно восстанавливают точку, которая принадлежит всем плоскостям. \\
        Первая координата точки --- секрет.
    \end{itemize}
\end{frame}

\begin{frame}{Схема Блэкли}
    Визуализация для \(k = 3\):
    \begin{figure}
    \centering
    \includegraphics[width=\linewidth]{blakely_scheme.png}
\end{figure}
\end{frame}

\begin{frame}{Схема Блэкли. Пример}
    Реализована \((n, 3)\)- пороговая СРС Блэкли. \\
    Уравнение Алисы: \(2x + 3y + z - 17 = 0\) \\
    Уравнение Боба: \(x - y + 4z - 15 = 0\) \\
    Уравнение Евы: \(5x + 2y - z - 12 = 0\) \\
    Восстановление точки: \[
    \begin{cases}
        2x + 3y + z - 17 = 0 \\
        x - y + 4z - 15 = 0 \\
        5x + 2y - z - 12 = 0
    \end{cases} \longrightarrow \begin{pmatrix}
        2 & 3 & 1 & \vline & 17 \\
        1 & -1 & 4 & \vline & 15 \\
        5 & 2 & -1 & \vline & 12
    \end{pmatrix} \longrightarrow 
    \]
    \[
    \longrightarrow \begin{pmatrix}
        1 & 0 & 0 & \vline & 2 \\
        0 & 1 & 0 & \vline & 3 \\
        0 & 0 & 1 & \vline & 4
    \end{pmatrix}
    \]
    Секрет --- первая координата точки, т. е. 2.
\end{frame}

\begin{frame}{Схема Блэкли}
    \begin{Note}
        Схема Блэкли не является идеальной СРС, так как \(\forall i = \overline{1, n} \ \ |S_i| = |\mathbb{F}_q|^{k + 1} \neq |\mathbb{F}_q| = |S_0|\).
    \end{Note}
\end{frame}

\begin{frame}{Схэма Блэкли}
    Обозначим \(
         h_0 = (1, \dots, 0) \in \mathbb{F}_q^k, \ 
         h_i = (a_{i_1}, \dots, a_{i_k}) \ i = \overline{1, n}\). \\ \ \\
    Для того, чтобы СРС была совершенной дилер должен следить за следующими условиями:
    \begin{itemize}
        \item Любые \(k\) векторов \(h_{i_i}, \dots, h_{i_{k}}\) должны быть ЛНЗ. \\
        Необходимо для того, чтобы по любым \(k\) восстановить секрет.
        \item Для любых \(k - 1\) векторов \(h_{i_i}, \dots, h_{i_{k - 1}}\) вектор \(h_0\) не должен лежать в \(\langle h_{i_i}, \dots, h_{i_{k - 1}} \rangle\). \\
        Необходимо для того, чтобы любые \(k - 1\) участников не получали никакой дополнительной информации о секрете.
    \end{itemize}
\end{frame}

\begin{frame}{Схема Блэкли}
    \begin{claim}
        Схэма Блэкли является совершенной СРС.
    \end{claim}
    \begin{itemize}
        \item \(P(\xi_0 = x \ \vline \ \xi_{i_1} = s_{i_1}, \dots, \xi_{i_k} = s_{i_k}) \in \{ 0, 1 \} \); \\
        \item \(P(\xi_0 = x \ \vline \ \xi_{i_1} = s_{i_1}, \dots, \xi_{i_t} = s_{i_t}) = P(\xi_0 = x) \) при \(t < k\).
    \end{itemize}
\end{frame}

\begin{frame}{Схема Блэкли}
    \begin{itemize}
        \item \(P(\xi_0 = x \ \vline \ \xi_{i_1} = s_{i_1}, \dots, \xi_{i_k} = s_{i_k}) \in \{ 0, 1 \} \).
    \end{itemize}
    Запишем СЛАУ:
    \[
    \begin{cases}
        a_{i_1_1}x_1 + a_{i_1_2}x_2 + \dots + a_{i_1_k}x_{k} + d_{i_1} = 0 \\
        \vdots \\
        a_{i_k_1}x_1 + a_{i_k_2}x_2 + \dots + a_{i_k_k}x_{k} + d_{i_k} = 0 \\
    \end{cases}
    \]
    Так как все строки ЛНЗ, то существует решение и ровно одно. Это решение \((s_0, b_1, \dots, b_{k - 1})\). Тогда 
    \[
    P(\xi_0 = x \ \vline \ \xi_{i_1} = s_{i_1}, \dots, \xi_{i_k} = s_{i_k}) = P(\xi_o = x \ \vline \ \xi_0 = s_0) =
    \]
    \[
    = P(x = s_0) \in \{0, 1\}
    \]
\end{frame}

\begin{frame}{Схема Блэкли}
    \begin{itemize}
        \item \(P(\xi_0 = x \ \vline \ \xi_{i_1} = s_{i_1}, \dots, \xi_{i_t} = s_{i_t}) = P(\xi_0 = x) \) при \(t < k\).
    \end{itemize}
    Запишем СЛАУ в матричном виде:
    \[
    \resizebox{4cm}{1cm}{\left( \begin{array}{| c |}
        \hline \\
        h_{i_1} \\
        \hline \quad \quad \quad \quad \quad \quad\\
        \vdots \\
        \hline \\
        h_{i_{t}} \\
        \hline
    \end{array} \right) \begin{pmatrix}
        x_1 \\ \vdots \\ x_k
    \end{pmatrix} = \begin{pmatrix}
        -d_{i_1} \\ \vdots \\ -d_{i_k}
    \end{pmatrix}
    }
    \]
    Допишем уравнение для \(h_0\) с соответствующим значением \(x\):
    \[
    \resizebox{4cm}{1.3cm}{\left( \begin{array}{| c |}
        \hline \\ 
        h_0 \\
        \hline \\
        h_{i_1} \\
        \hline \quad \quad \quad \quad \quad \quad\\
        \vdots \\
        \hline \\
        h_{i_{t}} \\
        \hline
    \end{array} \right) \begin{pmatrix}
        x_1 \\ \vdots \\ x_k
    \end{pmatrix} = \begin{pmatrix}
        x \\ -d_{i_1} \\ \vdots \\ -d_{i_k}
    \end{pmatrix}
    }
    \]
\end{frame}

\begin{frame}{Схема Блэкли}
\[
    \resizebox{4cm}{1.3cm}{\left( \begin{array}{| c |}
        \hline \\ 
        h_0 \\
        \hline \\
        h_{i_1} \\
        \hline \quad \quad \quad \quad \quad \quad\\
        \vdots \\
        \hline \\
        h_{i_{t}} \\
        \hline
    \end{array} \right) \begin{pmatrix}
        x_1 \\ \vdots \\ x_k
    \end{pmatrix} = \begin{pmatrix}
        x \\ -d_{i_1} \\ \vdots \\ -d_{i_k}
    \end{pmatrix}
    }
    \]
    Так как \((h_0, h_{i_1}, \dots, h_{i_{t})}\) --- ЛНЗ, то система совместна для любого \(x\). \\
    При этом для любого \(x\) количество решений неоднородной СЛАУ равно количеству решений ОСЛАУ, то есть не зависит от \(x\). То есть распределение секрета не изменилось.
\end{frame}

\begin{frame}{Обобщение}
    Выпишем в матричном виде системы для схемы Шамира \\ и для схемы Блэкли:
    \[
    \resizebox{5cm}{1cm}{ \begin{pmatrix}
        1 & 0 & \dots & 0 \\
        1 & x_1 & \dots & x_1^{k - 1} \\
        \vdots & \vdots & \ddots & \vdots \\
        1 & x_n & \dots & x_n^{k - 1}
    \end{pmatrix} \begin{pmatrix}
        a_0 \\ \vdots \\ a_{k - 1}
    \end{pmatrix} = \begin{pmatrix}
        s_0 \\ \vdots \\ s_n
    \end{pmatrix}
    } \quad \quad \quad \quad 
    \resizebox{5cm}{1cm}{ \begin{pmatrix}
        1 & 0 & \dots & 0 \\
        a_{1_1} & a_{1_2} & \dots & a_{1_k} \\
        \vdots & \vdots & \ddots & \vdots \\
        a_{n_1} & a_{n_2} & \dots & a_{n_k} \\
    \end{pmatrix} \begin{pmatrix}
        x_1 \\ \vdots \\ x_k
    \end{pmatrix} = \begin{pmatrix}
        s_0 \\ -d_{1_1} \\ \vdots \\ -d_{n_k}
    \end{pmatrix}
    }
    \]
    Видно, что во многом эти схемы схожи, у них отличается только матрица. \\
    Давайте обозначим матрицу \(H\) с векторами строками \(h_0, h_1, \dots, h_n\), \\
    неизвестный вектор столбец -- \(x\), \\
    известный вектор столбец -- \(s\). \\
    Тогда оба этих уравнения переписываются в виде \(Hx = s\).
\end{frame}

\begin{frame}{Линейная СРС}
    \(Hx = s\) \\
    А что будет, если в качестве матрицы \(H\) взять какую-то случайную? \\
    \begin{define}
    Получим \textbf{линейную СРС}. В ней:
    \begin{itemize}
        \item множество индексов \(\{ i_1, \dots, i_m \} \in \mathcal{A}\) (является разрешенным), если \(h_0 \in \langle h_{i_1}, \dots, h_{i_m} \rangle \);
        \item множество индексов \(\{ i_1, \dots, i_l \} \not\in \mathcal{A}\) (не является разрешенным), если \(h_0 \not\in \langle h_{i_1}, \dots, h_{i_m} \rangle \);
        \item \(s_i\) --- доля \(i\)-го участника;
        \item \(s_i = (h_0, x)\) --- секрет.
    \end{itemize}
    \end{define}
    
    \begin{Note}
        Доказательство совершенности схэмы Блэкли обобщается на случай линейной СРС. Таким образом все линейные СРС совершенны.
    \end{Note}
\end{frame}

% Заключение
\section{Задачи}
\begin{frame}{Задачи}
    \begin{center}
        Вариант 1
    \end{center}
    \begin{enumerate}
        \item Выступая в роли дилера, разделите секрет \(s_0 = 22\) в \(\mathbb{F}_{29}\), реализовав \((5, 3)\)-пороговую СРС с помощью
        \begin{enumerate}
            \item алгоритма Шамира
            \item алгоритма Блэкли
        \end{enumerate}
        \item Реализована \((n, 3)\)-пороговая СРС Шамира в \(\mathbb{F}_{29}\). \\
        Алисе соответсвует число 1, Бобу --- 2, Еве --- 3. \\
        Доля Алисы --- 7, доля Боба --- 26, доля Евы --- 11. \\
        Воспроизведите процесс восстановления секрета Алисой, Бобом и Евой.
    \end{enumerate}
\end{frame}

\section{Задачи}
\begin{frame}{Задачи}
    \begin{center}
        Вариант 2
    \end{center}
    \begin{enumerate}
        \item Выступая в роли дилера, разделите секрет \(s_0 = 7\) в \(\mathbb{F}_{29}\), реализовав \((5, 3)\)-пороговую СРС с помощью
        \begin{enumerate}
            \item алгоритма Шамира
            \item алгоритма Блэкли
        \end{enumerate}
        \item Реализована \((n, 3)\)-пороговая СРС Шамира в \(\mathbb{F}_{29}\). \\
        Алисе соответсвует число 1, Бобу --- 2, Еве --- 3. \\
        Доля Алисы --- 9, доля Боба --- 3, доля Евы --- 23. \\
        Воспроизведите процесс восстановления секрета Алисой, Бобом и Евой.
    \end{enumerate}
\end{frame}

% Последний слайд
\begin{frame}
    \centering
    \Huge Спасибо за внимание!
\end{frame}

\end{document}

